\documentclass[a4paper, 10pt, conference]{IEEEtran}
\IEEEoverridecommandlockouts
% The preceding line is only needed to identify funding in the first footnote. If that is unneeded, please comment it out.
\usepackage{cite}
\usepackage{amsmath,amssymb,amsfonts}
\usepackage{algorithmic}
\usepackage{graphicx}
\usepackage{textcomp}
\addtolength{\topmargin}{0cm}
\addtolength{\textheight}{0.7in}
\usepackage{xcolor}
\def\BibTeX{{\rm B\kern-.05em{\sc i\kern-.025em b}\kern-.08em
    T\kern-.1667em\lower.7ex\hbox{E}\kern-.125emX}}
\begin{document}

\title{Inteligência Artificial Aplicada ao Controle de Semáforos para Otimização de Fluxo Veicular e Pedestre*\\
\thanks{Trabalho realizado por Matheus Corteletti Delfino e orientado pelo Prof. Gustavo Maia no Instituto Federal do Espírito Santo, Campus Serra, curso de Engenharia de Controle e Automação, na disciplina de Inteligência Artificial.}
}

\author{\IEEEauthorblockN{Matheus Corteletti Delfino}
\IEEEauthorblockA{\textit{Instituto Federal do Espírito Santo, Campus Serra} \\
\textit{Engenharia de Controle e Automação}\\
Serra, Brasil \\
matheusrlz@hotmail.com}
\and
\IEEEauthorblockN{Prof. Gustavo Maia}
\IEEEauthorblockA{\textit{Instituto Federal do Espírito Santo, Campus Serra} \\
\textit{Engenharia de Controle e Automação}\\
Serra, Brasil \\
gustavo.maia@ifes.edu.br}
}

\maketitle


\begin{abstract}
O crescente desafio do tráfego urbano requer soluções inovadoras para aprimorar a eficiência dos semáforos. Neste artigo, é proposto uma abordagem fundamentada em inteligência artificial para a otimização dos tempos de sinal verde, empregando o algoritmo de agrupamento KMeans.
\end{abstract}

\begin{IEEEkeywords}
Controle, Mobilidade Urbana, Otimização, Simulação , KMeans
\end{IEEEkeywords}

\section{Introdução}
O crescente congestionamento no tráfego urbano representa um desafio constante, exigindo soluções inovadoras para melhorar a fluidez viária. Um dos pontos críticos nesse cenário é a gestão dos semáforos, responsáveis por regular o fluxo de veículos e pedestres. Nesse contexto, surge a necessidade de uma abordagem eficaz para otimizar os tempos de verde dos semáforos, levando em consideração variáveis dinâmicas como o volume de veículos e pedestres.

Este artigo propõe uma solução baseada em inteligência artificial, utilizando o algoritmo de agrupamento KMeans, para aprimorar a eficiência dos semáforos urbanos. A ideia central é estabelecer uma conexão entre os semáforos distribuídos pela cidade e um servidor centralizado de controle de tempos. Dessa forma, o KMeans entra em cena para aprender padrões a partir dos dados coletados, permitindo a adaptação dinâmica dos tempos de verde com base nas condições reais do tráfego.

Ao conectar os semáforos a um servidor central, é possível implementar uma abordagem mais inteligente e adaptativa, contribuindo significativamente para a melhoria do fluxo de tráfego nas áreas urbanas.


\section{Objetivo do Projeto}

O principal objetivo deste projeto é conceber e implementar um sistema de controle de tráfego adaptativo, considerando não apenas a quantidade de veículos e pedestres, mas também as particularidades de cada cruzamento. A proposta visa criar uma solução inovadora que otimize a fluidez do tráfego urbano, reduzindo significativamente os tempos de espera nos semáforos.

Almejamos desenvolver um sistema inteligente e dinâmico que possa se adaptar em tempo real às condições variáveis do tráfego, utilizando algoritmos de inteligência artificial, especialmente o KMeans. A ideia é proporcionar uma experiência de deslocamento mais eficiente para os usuários das vias urbanas, minimizando congestionamentos e promovendo uma mobilidade urbana mais fluida e sustentável.


\section{Metodologia}

A metodologia adotada fundamenta-se em uma abordagem de aprendizado de máquina, tendo o algoritmo KMeans como peça central para agrupar cruzamentos semelhantes. Para alcançar esse propósito, incorporamos variáveis cruciais, como a quantidade de pedestres, veículos e características específicas da via.

O processo se inicia com a coleta e preparação dos dados dos semáforos, incluindo informações sobre a infraestrutura viária e o fluxo de tráfego. Em seguida, aplicamos o algoritmo KMeans para realizar o agrupamento desses cruzamentos, considerando suas similaridades. O modelo resultante é integrado a um simulador de tráfego, permitindo a avaliação do desempenho do sistema em condições diversas.

A escolha do KMeans como técnica de agrupamento visa proporcionar uma adaptação dinâmica dos tempos de semáforo com base em padrões identificados, contribuindo para uma gestão eficiente do tráfego urbano. Ao utilizar um simulador, podemos simular cenários variados e aprimorar o modelo de controle de tráfego adaptativo, visando melhorar a fluidez e reduzir os tempos de espera nos semáforos.


\section{Desenvolvimento}

Nesta seção, abordaremos detalhes relevantes sobre o desenvolvimento do projeto, desde a aplicação console utilizada para a simulação até a estrutura do projeto e os comandos essenciais para utilização.

\subsection{Aplicação Console e Uso do Python}

Para a simulação, utilizamos uma aplicação console desenvolvida em Python. A estrutura do projeto segue as boas práticas de organização, proporcionando uma fácil compreensão e manutenção. Para instalar a aplicação, execute o seguinte comando:

\begin{verbatim}
pip install https://github.com/Batchuka/
PoC-TraficControl-IA/raw/master/dist/
pocKmeansTrafficControl-1.0.tar.gz
\end{verbatim}

Uma vez instalado, inicie a simulação com o seguinte comando:

\begin{verbatim}
python -m pocKmeansTrafficControl
\end{verbatim}

Este comando ativa a aplicação console e inicia a simulação do controle de tráfego adaptativo baseado em KMeans.

Os dados são gerados aleatoriamente, com atenção especial para os pesos atribuídos a certas características. O mapeamento de categorias, como "residencial," para pesos específicos é realizado para refletir a influência dessas características no modelo.




\subsection{Decisão dos Dados para o Modelo}

A estrutura dos dados segue um formato JSON, incluindo informações cruciais para a simulação do controle de tráfego adaptativo. Abaixo, descrevemos cada atributo presente nos dados:

\begin{itemize}
  \item \textbf{"id\_semaforo"}: Identificação única do semáforo.
  \item \textbf{"qld\_via"}: Categoria da via, podendo ser "residencial," "comercial," ou "industrial."
  \item \textbf{"qld\_porte"}: Porte da via, representado por um valor numérico.
  \item \textbf{"qtd\_pedestres"}: Quantidade de pedestres presentes na área do semáforo.
  \item \textbf{"qtd\_veiculos"}: Quantidade de veículos presentes na área do semáforo.
  \item \textbf{"tempo\_verde"}: Tempo de verde inicialmente definido para o semáforo.
\end{itemize}

Esses dados são gerados aleatoriamente para simular cenários variados. A quantidade de pedestres e veículos, bem como a categoria da via e seu porte, são consideradas de forma ponderada para refletir a diversidade nas condições do tráfego. O tempo de verde inicial é atribuído de acordo com o agrupamento realizado pelo modelo KMeans durante o treinamento.

\subsection{Atualização dos Semáforos}

A classe responsável pelos semáforos gerencia a atualização de seus dados em uma simulação que envolve 50 semáforos. Após a criação e salvamento do modelo KMeans, os dados dos semáforos são ligeiramente modificados para simular mudanças na dinâmica do tráfego. Essa alteração leve nos dados reflete as variações normais que ocorrem nas condições de tráfego ao longo do tempo.

Essa abordagem de atualização contínua dos semáforos permite avaliar como o modelo KMeans reage e adapta seus tempos de verde diante de mudanças nas condições do tráfego, contribuindo para a eficácia do controle de tráfego adaptativo.


\subsection{Construção do Modelo}

A construção do modelo KMeans é uma etapa crucial do projeto, onde variáveis relevantes dos semáforos são utilizadas para agrupar cruzamentos similares. Essas variáveis incluem a quantidade de veículos, pedestres, bem como características específicas da via, como porte e qualidade.

Durante a construção do modelo, são identificados clusters que representam grupos semelhantes de semáforos com base em suas características. A representação gráfica desses clusters pode ser visualizada através de um comando específico de plot, proporcionando insights visuais sobre a eficácia do agrupamento realizado pelo algoritmo KMeans.

A análise visual dos clusters formados é essencial para validar a capacidade do modelo em identificar padrões e similaridades entre diferentes cruzamentos. Essa representação gráfica oferece uma visão intuitiva da segmentação do espaço de características, sendo uma ferramenta valiosa na avaliação do desempenho do modelo KMeans no contexto do controle de tráfego adaptativo.


\subsection{Simulação}

A simulação é uma etapa crucial para testar e avaliar o sistema de controle de tráfego adaptativo baseado no modelo KMeans. A seguir, apresentamos uma sequência lógica que ilustra o processo:

\begin{itemize}
  \item \textbf{Instalação da Aplicação Console:}
  A primeira etapa consiste na instalação da aplicação console, que pode ser realizada através do seguinte comando:
  \begin{verbatim}
  pip install https://github.com/Batchuka/
  PoC-TraficControl-IA/raw/master/dist/
  pocKmeansTrafficControl-1.0.tar.gz
  \end{verbatim}

  \item \textbf{Início da Simulação:}
  Após a instalação, a simulação é iniciada através do seguinte comando:
  \begin{verbatim}
  python -m pocKmeansTrafficControl
  \end{verbatim}
  Esse comando inicia a aplicação console e inicia a simulação do controle de tráfego adaptativo baseado em KMeans.

  \item \textbf{Geração Aleatória de Dados:}
  Durante a simulação, os dados dos semáforos são gerados aleatoriamente, representando diferentes situações de tráfego.

  \item \textbf{Construção do Modelo KMeans:}
  Em seguida, o modelo KMeans é construído considerando as características dos semáforos e agrupando-os em clusters.

  \item \textbf{Atualização dos Semáforos:}
  Os dados dos semáforos são ligeiramente modificados, simulando mudanças nas condições de tráfego, e o modelo KMeans é reaplicado para atualizar as classificações dos semáforos nos clusters.

  \item \textbf{Visualização dos Resultados:}
  A aplicação console exibe resultados detalhados, incluindo informações sobre a eficácia do controle de tráfego adaptativo. Gráficos são gerados, mostrando a distribuição dos semáforos nos clusters e proporcionando insights visuais sobre a segmentação realizada pelo modelo KMeans.

  \item \textbf{Continuação da Simulação:}
  A simulação continua após fechar a janela de visualização dos resultados, permitindo a observação dinâmica das mudanças nos tempos de verde dos semáforos e a adaptação do sistema de controle de tráfego.
\end{itemize}



\section{Resultados e Discussão}

Após a realização da prova de conceito (PoC), os resultados obtidos demonstram a viabilidade do emprego do modelo KMeans no controle adaptativo de semáforos. No entanto, é essencial considerar que este trabalho é uma PoC e, portanto, deve ser tratado como um ponto de partida para estudos mais aprofundados.

A implementação do controle de tráfego adaptativo mostra-se promissora, indicando que a aplicação de aprendizado de máquina, especialmente o algoritmo KMeans, pode ser uma abordagem eficaz para otimizar os tempos de verde em semáforos urbanos.

\subsection{Considerações de Segurança}

No contexto de implementação em escala real, é fundamental destacar a importância de medidas de segurança robustas. Semáforos conectados em rede, especialmente através de Ethernet, devem ser protegidos contra possíveis ataques cibernéticos. A implementação prática desse sistema exige uma abordagem cuidadosa para garantir a integridade e segurança dos semáforos e da rede.

\subsection{Exploração de Modelos Mais Elaborados}

Embora o KMeans tenha se mostrado eficaz nesta PoC, trabalhos futuros podem explorar modelos mais elaborados de controle de tráfego. Modelos avançados que levam em consideração mais variáveis e dinâmicas complexas do tráfego podem ser investigados para melhorar ainda mais a eficiência do controle adaptativo.

\subsection{Trabalhos Posteriores}

Para trabalhos futuros, há diversas áreas a serem exploradas e aprimoradas. Uma consideração importante é a correlação entre as cores atribuídas pelo Matplotlib e os clusters gerados pelo KMeans. Garantir que as cores representem visualmente os diferentes clusters pode contribuir para uma melhor interpretação dos resultados.

Outro ponto de interesse é o trabalho disponível no repositório GitHub \footnote{https://github.com/BilHim/trafficSimulator/tree/main}, que apresenta um simulador visual de tráfego. Explorar a integração deste simulador com o modelo KMeans desenvolvido neste projeto pode proporcionar uma visualização mais detalhada e intuitiva do controle de tráfego adaptativo.

Em síntese, este trabalho serve como ponto de partida para futuras pesquisas e desenvolvimentos na área de controle de tráfego urbano, abrindo espaço para a implementação de soluções mais avançadas e eficazes.


\section*{Conclusão}

Este trabalho apresentou uma prova de conceito (PoC) para o uso do algoritmo KMeans no controle adaptativo de semáforos urbanos. Ao longo do desenvolvimento, foi possível construir e integrar um modelo KMeans que, baseado em variáveis como quantidade de veículos, pedestres e características da via, mostrou-se capaz de otimizar os tempos de verde dos semáforos.

Os resultados obtidos na simulação indicam a viabilidade dessa abordagem para melhorar a eficiência do controle de tráfego urbano. No entanto, é crucial reconhecer que este trabalho é uma PoC e, como tal, serve como ponto de partida para estudos mais aprofundados e implementações práticas.

Considerações de segurança são fundamentais ao considerar a implementação em escala real, especialmente quando os semáforos estão conectados em rede. Medidas robustas devem ser adotadas para proteger contra possíveis ameaças cibernéticas.

O desenvolvimento deste projeto abre caminho para trabalhos futuros, que podem explorar modelos mais elaborados, integrar simuladores visuais de tráfego e aprimorar a correlação entre os resultados do KMeans e representações visuais. Em última análise, este trabalho contribui para o avanço do campo de controle de tráfego urbano, proporcionando insights e perspectivas valiosas para pesquisadores e profissionais da área.

\section*{} % Isso remove o título da seção

\begin{thebibliography}{5}

\bibitem{pipis}
George Pipis. "How to Build Python Packages." Medium, Mar 8, 2023. \\ 
\textit{https://jorgepit-14189.medium.com/how-to-build-python-packages-e20f03fccabd}

\bibitem{arif}
Arif R. "Step by Step to Understanding K-means Clustering and Implementation with sklearn." Data Folks Indonesia, Oct 4, 2020. \\
\textit{https://medium.com/data-folks-indonesia/step-by-step-to-understanding-k-means-clustering-and-implementation-with-sklearn-b55803f519d6}

\bibitem{scikit-learn}
scikit-learn Documentation. \\
\textit{https://scikit-learn.org/stable/modules/classes.html}

\bibitem{matplotlib}
Matplotlib Documentation. \\
\textit{https://matplotlib.org/stable/api/index.html}

\bibitem{pandas}
Pandas Documentation. \\
\textit{https://pandas.pydata.org/docs/reference/index.html}

\end{thebibliography}
\end{document}
